\documentclass{article}
\usepackage[utf8]{inputenc}

\title{Summary 1}
\author{Prasant Acharya}
\date{January $28^{th}$, 2020}

\begin{document}

\maketitle
Proctor and Gamble (P$\&$G) is one of the oldest companies in the United
States, founded in the 1830s. They deal with household consumer products
such as toilet paper, detergents, diapers, soaps, etc. Given the industry that
they are in, one would think that they would not have a use for high-performance
computing. However, this assumption would be wrong.\par

Many of the problems that Proctor and Gamble have to face are inherently
contradictory. For example, most people want toilet paper that is both strong
and soft. These are contradictory properties that need to be both fulfilled
to have a good product. Another example of this is with detergents. Most
people want a detergent that both removes stains, but does not strip the dye
out of already colored fabrics. This, of course, is another catch 22, because stains
are also dyes. Due to the difficulty of trying to make a product that fulfills
contradictory desires, they have to do a lot of testing, and most of their testing
is done using simulations.\par

Not only are the problems that P$\&$G has to solve hard to solve from a product development standpoint, but also from a product manufacturing standpoint. P$\&$G deal with fairly low costs, high volume products. For example, in order to
sell one billion dollars, they have to make two billion diapers. And they make
several million diapers in a day, and about a billion in a week. Because of the
high volume, and the rate at which they produce, they often need to run simulations to make sure their methods of production are efficient. For example,
they will run simulations on a bottle design to figure out if it will fall off the
assembly line.\par

Due to the uniquie problems caused by the industry Proctor and Gamble is
in, as well as the types of problems that they have to solve, its no surprise that
they make use of high preformance computing. The hardware that they use to
run their simulations tends not to be the highest end when it was built. Usually,
their super-computers will lag about a decade from the best of the best.\par

They not only utalize simulations to test new products, but they will also
use simulations to find out where a product failed after they launched. For
example, they might run a simulation on a falling plastic bottle, or a razor to
figure out where it failed, and how to prevent it in the future.

\end{document}
