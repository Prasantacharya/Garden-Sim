\documentclass{article}
\usepackage[utf8]{inputenc}

\title{Summary 1}
\author{Prasant Acharya}
\date{January $28^{th}$, 2020}

\begin{document}

\maketitle
Proctor and Gamble (P$\&$G) is one of the oldest companies in the United States, founded in the 1830's. They deal with house hold consumer products such as toilet paper, detergents, diapers, soaps, etc. Given the industry that they are in, one would think that they would not have a use for high preformance computing. However, Proctor and Gamble utilize \par
Many of the problems that Proctor and Gamble have to face are inheriently contradictory. For example, most people want toilet paper that is both strong and soft. These are contradictory properties that need to be both fullfilled to have a good product. Another example of this is with detergents. Most people want a detergent that both removes stains, but does not strip the dye out of already colored fabrics. This of course is another catch 22, because stains are also dyes. Due to the \par
Not only are the problems that P$\&$G have to solve hard to solve and \par
Due to the uniquie problems caused by the industry Proctor and Gamble is in, as well as the types of problems that they have to solve, its no surprise that they make use of high preformance computing. The hardware that they use to run their simulations tends not to be the highest end when it was built. Usually, their super-computers will lag about a decade from the best of the best.\par
They not only utalize simulations to

\end{document}
